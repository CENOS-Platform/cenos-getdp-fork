% $Id: fem.tex,v 1.3 2001-06-13 17:18:12 geuzaine Exp $

% ---------------------------------------------------------------------------

\begin{slide}

\slidepagestyle{none}

\begin{center}
\bigtitle{Introduction --- finite element methods}
\ifnum\fulltitle=1\par\bigskip\bigskip
\mediumtitle{Christophe Geuzaine}\\
\bigskip
\smalltitle{Department of Electrical Engineering}\\
\smalltitle{Montefiore Institute B28, Sart Tilman Campus}\\
\smalltitle{University of Li�ge}\\
\smalltitle{B-4000 Li�ge (BELGIUM)}
\fi
\end{center}

\end{slide}

% ---------------------------------------------------------------------------
\part{Introduction}
% ---------------------------------------------------------------------------

\chapter{Coupled electromagnetic problems?}

\begin{slide}

Maxwell's equations, coupled with...

\begin{slideitemize}
\item Electric and electronic \emph{circuits} (power electronic supplies)
\item \emph{Mechanical} phenomena (force calculation, magnetostriction,
piezoelectricity, noise and vibrations)
\item \emph{Thermal} phenomena (thermal losses, induction heating, dielectric heating)
\item \emph{Fluid} dynamics (charged particules, magnetohydrodynamics)
\end{slideitemize}

\end{slide}

% ---------------------------------------------------------------------------

\chapter{Computational methods?}

\begin{slide}

\begin{slideitemize}
\item \emph{Analytic} models are difficult/impossible to apply to complex/coupled problems
\item \emph{Performance} (both floating point and visualization) of low end PCs is
exploding
\item Basic \emph{theory} of classic numerical methods (finite differences, finite
volumes, finite elements, integral methods) is now well known, and future
developments don't change the fundamental principles anymore
\end{slideitemize}

\end{slide}

% ---------------------------------------------------------------------------

\chapter{Finite element method (FEM)}

\begin{slide}

\begin{slideitemize}
\item 1960s for mechanical problems (very large \emph{application range} since 1980s)
\item Strong \emph{mathematical foundations} (convergence, unicity)
\item Generalizations/reinterpretations (vanishing boundaries between finite
differences, finite elements and finite volumes, ...) call for a
\emph{single sofware implementation}
\end{slideitemize}

But FEM is not the magic/universal tool:
\begin{slideitemize}
\item Many conflicting/\emph{antinomic} issues (continuous
vs. discontinuous, conform vs. non conform meshes, implicit vs. explicit,
...)
\item Generality always has a price (i.e.\ \emph{efficiency} trade-off)
\end{slideitemize}

\end{slide}

\begin{slide}

Based on a double \emph{discretization}: ``replace'' the
\begin{slideitemize}
\item function spaces to which the fields belong (e.g.\ $\Hone{\Omega}$,
$\Hcurl{\Omega}$, $\Hdiv{\Omega}$ and $\Ltwo{\Omega}$) by \emph{finite
dimensional function spaces}
\item domains on which these subspaces are defined by a union of elementary
geometrical elements of simple shapes (a ``\emph{mesh}'' or ``grid'')
\end{slideitemize}

\bigskip

\mybox{colbox}{\textwidth}{
\begin{center}
FEM\\ $\Updownarrow$\\ the finite dimensional subspaces are built so that
their bases are piecewise defined on the mesh
\end{center}
}

\end{slide}

\begin{slide}

One way to obtain a consistent Galerkin FEM formulation:
\begin{slideitemize}
\item Write a \emph{weak formulation} of the problem:

\begin{equation*}
\begin{cases}
L u = f \text{ in } \Omega \\
B u = g \text{ in } \Gamma 
\end{cases}
\Rightarrow\quad
\ivol[_\Omega]{u}{L^* v} - 
\ivol[_\Omega]{f}{v} + 
\int\limits_\Gamma Q_g(v) \, ds , 
\quad\forall v \in V(\Omega) 
\end{equation*}

% $L$ is a differential operator of order $n$ defined on $\Omega$
% 
% L^* is the adjoint of L:
%
% \ivol[_\Omega]{L u}{v} - \ivol[_\Omega]{u}{L^* v} = \int\limits_\Gamma Q(u,v) ds ,
%
% $Q$ is a bilinear function of $u$ and $v$ and in their derivatives up
% to the order $n-1$
%
% $Q_g$ is a linear form in $v$ which depends of $g$

\item Discretize with \emph{Whitney/mixed} elements $w_i$:
\begin{equation*}
\bar{u}, \bar{v} \in W(\Omega) , \quad
W(\Omega) = \text{span}\{w_i\} , \quad
W(\Omega) \subset V(\Omega)
\end{equation*}

% \bar{u} = \sum_{i} u_i w_i 

\end{slideitemize}

\end{slide}

\begin{slide}

\begin{slideitemize}
\item ``\emph{nodal}'' elements for ``0-forms'' (\emph{continuous} scalar
fields like scalar potentials, temperature, pressure, ...)

\item ``\emph{edge}'' elements for ``1-forms'' (vector fields with
\emph{continuous tangential components} across material interfaces, like
electric and magnetic fields, magnetic vector potential, ...)

\item ``\emph{facet}'' elements for ``2-forms'' (vector fields with
\emph{continuous normal components} across material interfaces, like
magnetic flux density, current density, ...)

\item ``\emph{volume}'' elements for ``3-forms'' (\emph{piecewise
continuous} scalar fields like charge density, heat source density, ...)

\end{slideitemize}

\end{slide}
