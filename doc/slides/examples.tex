% $Id: examples.tex,v 1.4 2001-06-14 14:48:32 geuzaine Exp $

% ---------------------------------------------------------------------------

\begin{slide}

\slidepagestyle{none}

\begin{center}
\bigtitle{Examples}\\
\ifnum\fulltitle=1\par\bigskip\bigskip
\mediumtitle{Patrick Dular and Christophe Geuzaine}\\
\bigskip
\smalltitle{Department of Electrical Engineering}\\
\smalltitle{Montefiore Institute B28, Sart Tilman Campus}\\
\smalltitle{University of Li�ge}\\
\smalltitle{B-4000 Li�ge (BELGIUM)}
\fi
\end{center}

\end{slide}

% ---------------------------------------------------------------------------
\part{GetDP/Gmsh examples}
% ---------------------------------------------------------------------------

\chapter{The simplest example: magnetostatics}

\begin{slide}

\mybox{colbox}{\textwidth}{
\begin{equation*}
\Curl{\vec{h}} = \vec{j} ,\quad
\Div{\vec{b}} = 0 \quad\text{and}\quad
\vec{b} = \mu \vec{h} + \mu_0 \vec{h}_m 
\end{equation*}
\begin{equation*}\label{eq:tonti-sta}
\begin{split}
\xymatrix{
 \color{colpos}\phi    \ar@{->}[r]^-{\GradSymb_h}  &
 \vec{h} \ar@{->}[r]^-{\CurlSymb_h} \ar@{<->}[d]^{\mu} &
 \vec{j} \ar@{->}[r]^-{\DivSymb_h}   &
 0 \\
 0       \ar@{<-}[r]^-{\DivSymb_e}&
 \vec{b} \ar@{<-}[r]^-{\CurlSymb_e}&
 \color{colpos}\vec{a} 
}
\end{split}
\end{equation*}
}

\begin{slideitemize}
\item Weak form of Gauss law: 
\begin{equation*}
%\ivol{\Div{\vec{b}}}{\phi'} = 0 \Rightarrow
\ivol{\vec{b}}{\Grad{\phi'}} + \isur{\psca{\vec{n}}{\vec{b}}}{\phi'} 
= 0
\quad \forall\phi'\in\Hone[_0]{\Omega}
\end{equation*}

\item Weak form of Ampere's law:
\begin{equation*}
%\ivol{\Curl{\vec{h}}}{\vec{a}'} = \ivol{\vec{j}}{\vec{a}'} \Rightarrow
\ivol{\vec{h}}{\Curl{\vec{a}'}} + \isur{\pvec{\vec{n}}{\vec{h}}}{\vec{a}'}
= \ivol{\vec{j}}{\vec{a}'} 
\quad \forall\vec{a}'\in\Hcurl[_0]{\Omega}
\end{equation*}

\end{slideitemize}

\end{slide}

\begin{slide}

\mybox{colbox}{\textwidth}{
\begin{center}
\emph{Scalar potential} formulation
\begin{equation*}\label{eq:hs+hr1}
\vec{h} = \vec{h}_s+\vec{h}_r ,\quad\text{with}\quad
\Curl{\vec{h}_s} = \vec{j}     \quad\text{and}\quad
\vec{h}_r = -\Grad{\phi}
\end{equation*}
\end{center}
}

\begin{equation*}
%\ivol{\Div{\vec{b}}}{\phi'} = 0 \Rightarrow
\ivol{\vec{b}}{\Grad{\phi'}} + \isur{\psca{\vec{n}}{\vec{b}}}{\phi'} 
= 0
\quad \forall\phi'\in\Hone[_0]{\Omega}
\end{equation*}

\bigskip
NB: choice of source field $\vec{h}_s$, tretament of multiply connected
$\Omega$, ...

\end{slide}

\begin{slide}

\mybox{colbox}{\textwidth}{
\begin{center}
\emph{Vector potential} formulation
\begin{equation*}\label{eq:hs+hr1}
\vec{b} = \Curl{\vec{a}}
\end{equation*}
\end{center}
}

\bigskip
NB: gauge for $\vec{a}$, ...

\end{slide}

% ---------------------------------------------------------------------------

\background{}{}{}

\ifx\pdfoutput\undefined

\chapter{Other examples...}

\begin{slide}

\begin{center}
\includegraphics[width=\textwidth]{picts/antenna1}
\includegraphics[width=\textwidth]{picts/antenna2}
\includegraphics[height=\textheight]{picts/indheat}
\includegraphics[angle=-90,width=\textwidth]{picts/line220kv}
\includegraphics[height=\textheight]{picts/magnet}
\includegraphics[height=\textheight]{picts/motoras}
\includegraphics[width=\textwidth]{picts/f16}

piezo-electricity, magnetostriction, non-homogeneous waveguides, photonic
cristals, electromagnetic shielding, dielectric heating, ...

\end{center}



\end{slide}

\else

\fi
